\documentclass[11pt]{article}
\usepackage[utf8]{inputenc}
\usepackage[utf8]{listings}
\title{\textbf{Zadanie II.9}}
\author{Damian Soliński\\
		Piotr Kulas\\
		Marek Skiba}
\date{04.12.2012}
\begin{document}

\maketitle

\section{Treść zadania}

Dla równania $f(x) = 0$, gdzie $f(x) = e^x + x + 2$, wczytać a,b $\in$ $\mathbf{R}$ takie, by $a<b$ oraz $f(a) * f(b) < 0$. Następnie, dopóki "użytkownik się nie znudzi", wczytywać wartość $0 < \epsilon < 1$ i metodą połowienia na $[a,b]$ przybliżyć z dokładnościa $\epsilon$ rozwiązanie tego równania. Rozwiązanie to przybliżyć również metodą Newtona z $x_0 = a$, przy czym $x_k$ będzie dobrym przybliżeniem, gdy $|x_k - x_(k-1)| \leq \epsilon$. Porównać ilość kroków wykonanych metodą połowienia i metodą Newtona.

\end{document}
